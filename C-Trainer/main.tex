\documentclass[a4paper, 12pt]{report}

%%%%%%%%%%%%
% Packages %
%%%%%%%%%%%%

\usepackage[english]{babel}
\usepackage[noheader]{sleek}
\usepackage{sleek-title}
\usepackage{sleek-theorems}
\usepackage{sleek-listings}

%%%%%%%%%%%%%%
% Title-page %
%%%%%%%%%%%%%%

\logo{./resources/img/dsb_logo.jpg}
\institute{Random University}
\faculty{Faculty of Whatever Sciences}
\department{Department of Anything but Psychology}
\title{A sleek \LaTeX{} template}
\subtitle{With a sleeker title-page}
\author{\textit{Author}\\François \textsc{Rozet}}
\supervisor{Linus \textsc{Torvalds}}
\context{Well, I was bored...}	
\date{\today}

%%%%%%%%%%%%%%%%
% Bibliography %
%%%%%%%%%%%%%%%%

% \addbibresource{./resources/bib/references.bib}

%%%%%%%%%%
% Others %
%%%%%%%%%%

\lstdefinestyle{latex}{
    language=TeX,
    style=default,
    %%%%%
    commentstyle=\ForestGreen,
    keywordstyle=\TrueBlue,
    stringstyle=\VeronicaPurple,
    emphstyle=\TrueBlue,
    %%%%%
    emph={LaTeX, usepackage, textit, textbf, textsc}
}

\FrameTBStyle{latex}

\def\tbs{\textbackslash}

%%%%%%%%%%%%
% Document %
%%%%%%%%%%%%

\begin{document}
    \maketitle
    \romantableofcontents

   
\chapter{Features}

\section{\texttt{sleek}}

There are three available options for the \texttt{sleek} package:

\begin{enumerate}[noitemsep]
	\item \texttt{parindent}: Adds indentation to the first line of paragraphs.
	\item \texttt{noheader}: Removes the document header.
	\item \texttt{french}: Changes the decimal sign to a comma and translates some captions.
\end{enumerate}


\subsection{Mathematics}

This template uses \texttt{amsmath} and \texttt{amssymb}, which are the de-facto standard for typesetting mathematics. Additionally, \texttt{esint} provides alternative integral symbols (\emph{cf.} Table 78 in and \texttt{bm} is used for bold math symbols like vectors (see \eqref{eq:gauss_law}).

A few custom macros have also been added such as \texttt{\tbs{}rbk}, \texttt{\tbs{}sbk}, and \texttt{\tbs{}cbk} for respectively round, square, and curly brackets, \texttt{\tbs{}abs} for absolute value, \texttt{\tbs{}norm} for norm, \texttt{\tbs{}fact} for factorial and \texttt{\tbs{}diff} for up-right differential.
$$
\rbk{\frac{\pi}{2}}, \quad \sbk{\frac{\pi}{2}}, \quad \cbk{\frac{\pi}{2}}, \quad \abs{\frac{\pi}{2}}, \quad \norm{\frac{\pi}{2}}, \quad \fact{n} = \prod_{i = 1}^{n} i, \quad \frac{\diff \bm{x}}{\diff t} = \bm{v}
$$

\begin{equation}\label{eq:gauss_law}
	\oiint_S \bm{E} \cdot \diff \bm{s} = \iiint_V \frac{\rho}{\varepsilon_0} \diff V
\end{equation}

\begin{equation*}
	e = \sum_{n=0}^\infty \frac{1}{n!}
\end{equation*}

\begin{subequations}
	\begin{align}
		\frac{\diff x}{\diff t} & = \alpha x - \beta xy \\
		\frac{\diff y}{\diff t} & = \delta xy - \gamma y
	\end{align}
\end{subequations}

\begin{align*}
	\ln \abs{x} + C & = \int \frac{1}{x} \,\diff x \\
	\exp(x) & = \lim_{n \to \infty} \rbk{1 + \frac{x}{n}}^n
\end{align*}

\begin{equation}
	\left\{
	\begin{aligned}
		x & = r \sin \theta \cos \phi \\
		y & = r \sin \theta \sin \phi \\
		z & = r \cos \theta
	\end{aligned}
	\right.
\end{equation}

\begin{alignat*}{2}
	& P(A, B)  & = P(A \mid B) P(B)                        \\
	\Leftrightarrow \quad & P(A \mid B) & = \frac{P(A, B)}{P(B)}                 \\
	&          & = P(B \mid A) \frac{P(A)}{P(B)}
\end{alignat*}

\subsection{Units}

It is possible to write, both in text or math modes, numbers without units (\emph{e.g.} \num{1}, \num{1.0}, \num{-1}, \num{3.14159}, \num{e100}, $N_A = \num{6.022e23}$), units without quantity (\emph{e.g.} $\si{\joule} = \si{\newton\meter} = \si{\kilogram\meter\squared\per\second\squared}$) and, finally, quantities with their units (\emph{e.g.} \SI{9.81}{\meter\per\second\squared}, $c = \SI{299.6e6}{\meter\per\second}$).

\subsection{Lists}

Sleek Template uses \texttt{enumitem} to enhance the listing capabilities of \LaTeX{}. There are three lists environments:
\begin{enumerate}
	\item \texttt{itemize} for unordered lists;
	\item \texttt{enumerate} for ordered lists;
	\item \texttt{description} for descriptive lists.
\end{enumerate}

In a list, each element is preceded by the command \texttt{\tbs{}item}. It is possible to modify the labels
\begin{itemize}
	\item individually with \texttt{\tbs{}item[newLabel]};
	\item for the whole environment with the \texttt{label=newLabel} option.
\end{itemize}

One could want to reduce the space between items with the \texttt{noitemsep} option or to delete the left margin with the \texttt{leftmargin=*} option.

It is also possible to write nested lists. Here follows a very condensed example.

\subsection{Figures}

Thanks to the \texttt{graphicx} package, it is possible to include external graphic documents (images, plots, etc.) in your document with the \texttt{\tbs{}includegraphics} command. Most image type format (\texttt{jpg}, \texttt{png}, \texttt{bmp}, etc.) are supported by this command. However, it should be noted that it is highly preferable to use vectorial types, such as \texttt{pdf} or \texttt{eps}.

\begin{figure}[H]
	\centering
	\includegraphics[width=0.5\textwidth]{resources/pdf/logo.pdf}
	\noskipcaption{Random University logo.}
	\label{fig:random_university_logo}
\end{figure}

\subsection{Tables}

The packages \texttt{multicol} and \texttt{multirow} come in handy for complex table formatting such as multi-column or multi-row cells.

\begin{table}[H]
	\centering
	\begin{tabular}{|r|r|c|l|}
		\hline
		\multicolumn{3}{|l|}{a} & qrs  \\ \hline
		b &  ef &     jkl      & tuvx \\ \hline
		cd & ghi &     mnop     & wyz  \\ \hline
	\end{tabular}
	\caption{Example of multi-column cells.}
	\label{tab:multicol_example}
\end{table}

\begin{table}[H]
	\centering
	\begin{tabular}{|l|c|r|}
		\hline
		\multirow{3}{2cm}{a} &   b   &    c \\ \cline{2-3}
		&  de   &   fg \\ \cline{2-3}
		&  hij  &  klm \\ \hline
		nopq                 & rstuv & wxyz \\ \hline
	\end{tabular}
	\caption{Example of multi-row cells.}
	\label{tab:multirow_example}
\end{table}

\newpage
\section{\texttt{sleek-theorems}}

\texttt{sleek-theorems} is based on the \texttt{amsthm} and \texttt{thmtools} packages. It provides a handful of theorem-like environments, each of which has a different style and purpose.

The environments are \texttt{thm} (theorem), \texttt{lem} (lemma), \texttt{prop} (proposition), \texttt{proof}, \texttt{defn} (definition), \texttt{hyp} (hypothesis), \texttt{meth} (method), \texttt{quest} (question), \texttt{answ} (answer), \texttt{expl} (example), \texttt{rmk} (remark), \texttt{note}, and \texttt{tip}.

\begin{note}
	The option \texttt{french} translates the name of each provided environment. It is also possible, and easy, to add your own language as an option in the source code.
\end{note}

\begin{thm}[Triangle inequality]
	Let be a triangle in Euclidean space. Then the sum of the lengths of two of its sides always surpasses or equals the length of the third.
\end{thm}

\begin{proof}
	Let $a$, $b$, and $c$ be the lengths of the sides of a triangle in Euclidean space and $\alpha$, $\beta$, $\gamma$ their respective opposite angles. By the generalized Pythagoras' theorem, we have
	\begin{alignat*}{2}
		&  & c^2 & = a^2 + b^2 - 2ab \cos\gamma \\
		&  &     & \leq a^2 + b^2 + 2ab         \\
		&  &     & \leq (a + b)^2               \\
		\Leftrightarrow \quad &  & c   & \leq a + b
	\end{alignat*}
	Therefore, in any triangle, the sum of the lengths of two sides always surpasses or equals the length of the third.
\end{proof}

In addition, these environments also have framed versions -- \texttt{framedthm}, \texttt{framedlem}, etc. -- for readability.

\begin{framedthm}[Triangle inequality]\label{thm:Triangle inequality}
	Let be a triangle in Euclidean space. Then the sum of the lengths of two of its sides always surpasses or equals the length of the third.
\end{framedthm}

\begin{framedprf}
	Let $a$, $b$, and $c$ be the lengths of the sides of a triangle in Euclidean space and $\alpha$, $\beta$, $\gamma$ their respective opposite angles. By the generalized Pythagoras' theorem, we have
	\begin{alignat*}{2}
		&  & c^2 & = a^2 + b^2 - 2ab \cos\gamma \\
		&  &     & \leq a^2 + b^2 + 2ab         \\
		&  &     & \leq (a + b)^2               \\
		\Leftrightarrow \quad &  & c   & \leq a + b
	\end{alignat*}
	Therefore, in any triangle, the sum of the lengths of two sides always surpasses or equals the length of the third. \qedadd
\end{framedprf}

\begin{framedquest*}
	Based on the theorem \ref{thm:Triangle inequality}, what is the shortest path from a point $A$ to a point $B$ in Euclidean geometry?
\end{framedquest*}

   

%    \printbibliography


\end{document}
